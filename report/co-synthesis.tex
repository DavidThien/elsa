\section{Co-Synthesis} \label{sec:co-synthesis}
Because the behavior of functions only makes sense in the context of a
  fixed encoding, it is natural to want to specify the behavior of
  terms and functions in the grammar independently of any particular
  encoding.
Instead, we would like to be able to specify more abstract syntactic
  operations on environment terms, and have the synthesizer find both
  the terms' encodings, as well as the functions to go with those terms.
Neither the terms, nor the functions have any inherent meaning outside
  of a particular encoding, so the terms and functions \texttt{must} be
  described together.
We call synthesis from this description of the behavior of both terms
  and functions, \texttt{co-synthesis}.

Note that in the case of untyped lambda calculus, being able to specify
  the behavior in terms of a particular encoding, or in some higher-level
  specification, is necessary to synthesize terms.
In theory, any expression could be ascribed any value, but being able to
  synthesize the functions that operate how we want them to on those values
  requires that they have a certain structure which allows for that.
This means that there has to be some notion of what we want to do with
  the terms we are generating, if we want them to actually be useful.

\subsection{Search Process}
