\section{Co-Synthesis of Encodings} \label{sec:co-synthesis}
%
The functions described in the previous section only makes
sense in the context of a fixed encoding, but in general, a
data type accepts more than one.
%
As encodings are just expressions we would like to
automatically derive them as a result of synthesis.
%
To do so we first need to find a way to give an abstract
specification of the meaning of a data type and we need to
do it in a way that is independent of an underlying
encoding.
%
In the following, we see how we can extend the ideas
presented in \cref{sec:synthesis} to express these abstract
specifications in a natural way.

In general, our approach consist of specifying the behavior
of data types in relation with the functions that operates
on them.
%
We call this a \emph{co-specification} of a data type and
its functions.
%
We also refer to the process of synthesizing an encoding
together with the functions that operate on it as
\emph{co-synthesis}.

Our next key insight is that we can build natural and
concise co-specifications by looking at the introduction and
elimination forms of a data type.
%
Consider booleans for example.
%
Their introduction corresponds to the values \true and \false
and their elimination corresponds to case analysis using the
if-then-else (\f{ite}) function.
%
Using the ideas presented in \cref{sec:io} we can co-specify
\true, \false and \f{ite} as:
%
\begin{align*}
  \f{?ite}\;\f{?true}\;x\;y &\equiv x \\
  \f{?ite}\;\f{?false}\;x\;y &\equiv y
\end{align*}

The co-specification for natural numbers requires a little
bit more work.
%
First, we identify $0$ and \f{succ} as the introduction
forms of natural numbers in its usual inductive definition.
%
Second, we note that the elimination form of natural numbers
correspond to the ability of defining recursive functions on
them.
%
Instead of spelling out a general recursive principle we
specify the recursive behavior by picking a particular
recursive definition.
%
In our case we specify the elimination form in terms of the
recursive definition of the function \f{plus}.
%
We can now co-specify $0$, \f{succ} and \f{plus} as:
\begin{align*}
  \f{?plus}\;\f{?}0\;y &\equiv y\\
  \f{?plus}\;(\f{?succ}\;x)\;y &\equiv \f{?succ}\;(\f{?plus}\;x\;y)\\
\end{align*}

We conclude by showing how these ideas also apply to
non-recursive data types by showing a co-specification for pairs.
%
The introduction form corresponds to the constructor
\f{pair} and the elimination forms to the first and
second projections.
%
\begin{align*}
  \f{?first}\;(\f{?pair}\;x\;y) &\equiv x\\
  \f{?second}\;(\f{?pair}\;x\;y) &\equiv y\\
\end{align*}
