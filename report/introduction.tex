\section{Introduction} \label{sec:intro}
%
The untyped lambda calculus presents itself as an interesting
target for synthesis.
%
On one hand, it is described by a very simple grammar
defining a very unstructured program space for which it is
hard to guide a search.
%
On the other hand, this same lack of structure makes it
unclear what should be used as behavioral constraints.
%
In this work we explore the possibility of synthesizing
expressions in the untyped lambda calculus focusing mainly
in the second problem.

More specifically, we focus on the problem of synthesizing
functions over data encodings.
%
Data encodings in the lambda calculus are interesting
because the same basic construct, lambda abstractions, is
used to describe both functions and data.
%
This produces a tension which makes it hard to reason about
lambda expressions because the distinction between data and
function gets blurred.

We recognize this as one of the main problems that makes it
hard to specify behavioral constraints for lambda
expressions.
%
For this reason we note that in general it is not enough to
just specify the extensional behavior of functions and we
examine the possibility of constraining their evaluation.
%
One of our key insights is that using free variables give us
enough expressiveness to talk about the evaluation of
functions.

Based on this idea we note that by looking at the
introduction and elimination rules of data types we are able
to specify their meaning precisely by giving a specification
of their behavior in relation to some functions operating on
them.
%
We call this a \emph{co-specification} of a data type.
%
This opens the possibility of automatically deriving the
encoding for some data type as a product of synthesis in a
process we dubbed \emph{co-synthesis}.

Finally, we proved these ideas in practice by implementing a
prototype enumerative synthesizer.
%
Despite its naive implementation we were able to apply it in
practice to synthesize a series of interesting functions.
%
Perhaps most interestingly, our implementation was able to
automatically derive the standard encoding for booleans.

% ; the meaning for a function only exists
% in the context of a specific encoding of your chosen terms.


% This makes it hard to

% defining a very particular search space.
% %

% for two major reasons:
%   \begin{enumerate}
%     \item The grammar describing the language is very small, meaning that
%       small optimizations in the search process are very important to
%       constrain the search space as much as possible
%       %% TODO: This isn't quite the right description
%     \item Untyped lambda calculus uses the exact same grammar to describe
%       both terms, and functions; the meaning for a function only exists
%       in the context of a specific encoding of your chosen terms.
%   \end {enumerate}
% The second reason in particular is interesting, because it begs the
%   question of creating a specification for the behavior of functions
%   independently of a given encoding.
% This would allow users of the synthesizer to either pre-specify a
%   particular encoding of terms they are interested in synthesizing
%   functions for, or to specify the behavior of terms in the context of
%   function applications, and have the synthesizer synthesize both
%   an encoding for the terms, and the functions requested.
% We call synthesis of terms and functions from a single unified encoding
%   \emph{co-synthesis}.

% \subsection{Grammar}

% The grammar for untyped lambda calculus is simply defined as:
%   \begin{grammar}
%     <expr> ::= x
%       \alt <expr> <expr>
%       \alt $\lambda$ x <expr>
%   \end{grammar}
%   where $x$ is a variable.
% Note that variables need not necessarily be bound by the immediately
%   preceeding lambda to be referenced.
% For our purposes, variables only need be bound in the ultimate term
%   that we eventually produce, as we can assume without loss of
%   generality that there is no outside environment that we reference
%   for synthesized terms or functions.

% This grammar is very compact, which means that our search space tends to
%   blow up because of the necessary depth of the search, rather than
%   because the number of possible terms enumerable at a given depth
%   is large.

% \subsection{Evaluation Strategy}

% The meaning of a term in Lambda calculus, as in other programming languages,
%   depends on the evaluation strategy used to assign meaning to that term.
% For our purposes, we use normal order evaluation.
% Normal order evaluation evaluates the leftmost outtermost reducible
%   expression first.
% Although this will generate different terms than another strategy like
%   applicative order, there is nothing specialized about this synthesis
%   that requires normal order to be used above anything else.
% We have simply chosen normal order for consistency.
