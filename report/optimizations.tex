\section{A Enumerative Synthesizer} \label{sec:optimizations}
%
To test our ideas in practice we implemented a simple
enumerative synthesizer.
%
In the remainder of this section we detail some of the
challenges and optimizations we implemented.

\subsection{Beta-Normal Form Enumeration}
%
The synthesizer enumerates terms in beta-normal form.
%
This is probably the most important optimization as it
largely reduces the search space.
%
More specifically, our implementation makes a bottom up
enumeration of the expressions in the following grammar:

\begin{align*}
  e & ::= \lambda x. e ~|~ \lambda x. t \\
  t & ::= x ~|~ t\;t ~|~ t\;e
\end{align*}

\subsection{Number of Function Arguments}
One minor optimization that we were able to implement was to require
  functions to be enumerated in a form such that the first $n$ non-terminals
  were lambdas where $n$ is the number of arguments to the function.
For example, the function $\f{add}$ which takes two arguments must have the
  form $\lambda x. \lambda y. E$ where $E$ is the rest of the expression.
Although this is a minor optimization, it allowed us to effectively decrease
  the depth of our search space by the number of arguments to a function, and
  we would not have been able to synthesize $\f{add}$ without it.

\subsection{Enumerating Closed Terms}
%
One additional challenge in generating lambda expressions
lies in generating only closed terms; terms with no free
variables.
%
A nice trick we found makes this easier is to generate
variables in a separate step.
%
That is, we first enumerate lambda expressions were we leave
a ``hole'' in the place a variable should go and then fill
those holes with variables in scope which is easier once the
rest of the term is fixed.

We also found it easier to work with De Bruijn indices.
%
With explicit names, it becomes necessary for the
synthesizer to have some sort of variable context that it
passes around when generating variable terms, both so that
it knows what names to gives new bindings, as well as to
know which bindings it has available.
%
Passing through such a context introduces a lot of
complexity to the enumeration itself.
%
Using De Bruijn indices has the added benefit of removing
alpha renaming and alpha equivalence from our expression
evaluation.

  % In order to deal with this challenge, we exclusively use De
  % Bruijn indices to reference our variables.
  % This simplifies our enumeration greatly, as we no longer need to pass through
  % a context to know which variable names are bound, instead we can just keep
  % track of the depth of the expression.
