\section{Synthesizing Functions for a Given Encoding} \label{sec:synthesis}
%
As a first step, we tackle the problem of synthesizing
functions over a given encoding.
%
We begin with an overview of the standard encoding for the
data types which we focused our development on.
%
We then explore how to specify the behavior of functions for
their synthesis.

% The first step we took for synthesizing expressions was to synthesize
%   individual functions given an encoding.

\subsection{Church Encodings}
%
We tested and focused our development on three
different data types: \emph{booleans}, \emph{natural
numbers} and \emph{pairs}.
%
We give their standard encodings below.

\paragraph{Booleans}
%
The standard encoding for the boolean values \emph{true} and
\emph{false} is given by:
%
\begin{align*}
  \text{\true} &= \lambda a . \lambda b . a \\
  \text{\false} &= \lambda a . \lambda b . b
\end{align*}
%
Intuitively, boolean logic is interpreted as a choice and
boolean values are represented as functions of two
parameters making this choice: \true chooses its first
parameter and \false chooses its second parameter.

\paragraph{Natural numbers}
%
Church numerals are the standard representation of natural
numbers in the lambda calculus.
%
In this encoding the number $n$ is represented as a
higher-order function that maps a function $f$ to its
$n$-fold composition.
%
The value of the numeral is equivalent to the number of
times the function gets applied: 0 does not apply the
function at all, 1 applies the function once, 2 applies the
function twice, \emph{etc.}
%
\begin{align*}
  0 & = \lambda f . \lambda x . x \\
  1 & = \lambda f . \lambda x . f\,x \\
  2 & = \lambda f . \lambda x . f\,(f\,x) \\
  3 & = \lambda f . \lambda x . f\,(f\,(f\,x)) \\
  \vdots & \\
  n & = \lambda f . \lambda x . f^{\circ n}\,x
\end{align*}

\paragraph{Pairs}
%
A pair is represented as a function that takes two arguments
and returns a higher-order function which when provided with
a function as argument applies it to the two components of
the pair.
%
\begin{align*}
  \pair &= \lambda x. \lambda y. \lambda f.f\,x\,y
\end{align*}

\subsection{Synthesis from Examples} \label{sec:io}
%
We now turn to the question of how the behavior of a
function can be specified.
%
As it has proven useful in many scenarios, we explore the
idea of using input-output examples.

Consider, for instance, the boolean function \f{and}.
We can specify its behavior extensionally as follows:
%
\begin{align*}
\f{?and}\,\true\,\true    &\equiv \true \\
\f{?and}\,\true\,\false   &\equiv \false \\
\f{?and}\,\false\,\true   &\equiv \false \\
\f{?and}\,\false\, \false &\equiv \false \\
\end{align*}
%
By convention, we prefix the functions we wish to synthesize
with \f{?} and by convenience we refer to the boolean values
\true and \false by name, but they should be understood as
if their definition was expanded.
%
With this specification our tool can successfully find the
standard definition of \f{and} for church booleans:
%
\begin{align}
  \label{eq:and-def}
  \f{and} &= \lambda p. \lambda q. p\,q\,p
\end{align}

Similarly, we can apply this idea to numerals to specify the
behavior of the \f{plus} function.
%
In this case, however, we cannot specify the behavior on all
possible inputs but we can select a representative set.
%
\begin{align*}
  \f{?plus}\; 0\; 1 &\equiv 1\\
  \f{?plus}\; 1\; 0 &\equiv 1\\
  \f{?plus}\; 1\; 2 &\equiv 3\\
  \f{?plus}\; 2\; 2 &\equiv 4\\
\end{align*}
%
The above specification is sufficient for our tool to find
the definition:
%
\begin{align}
  \label{eq:plus-def}
  \f{plus} &= \lambda m. \lambda n. \lambda f. \lambda x. m\,f\,(n\,f\,x)
\end{align}



\subsection{Beyond Extensionality}
%
The specifications presented in the previous section only
spell out the extensional behavior of functions.
%
We found that it is also often useful to constraint the way
functions evaluate.
%
The ability to do so proved to be important for the
specification of encodings as we will see in
\cref{sec:co-synthesis}.

Recall the resulting encoding for \f{and} given in
\labelcref{eq:and-def}.
%
This definition uses the first argument as the test.
%
Under normal order evaluation, this corresponds to the usual
short-circuit evaluation where the second argument is only
evaluated when the first one is known to be false.
%
The purely extensional specification given in \cref{sec:io}
cannot distinguish this from a function that first
evaluates the second argument:
%
\begin{align}
  \label{eq:and-def2}
  \f{and} &= \lambda p. \lambda q. q\,p\,q
\end{align}
%
One of our main insights is that this sort of evaluation
behavior can be specified using free variables.
%
For example, the function \f{and} could be specified as:
%
\begin{align*}
  \f{?and}\,\true\,x &\equiv x \\
  \f{?and}\,\false\,x &\equiv \false \\
\end{align*}
%
Here, the variable $x$ is meant to remain free as opposed to
\true and \false which should be expanded to their
definition.
%
This specification effectively rule out
\labelcref{eq:and-def2} from the possible results.

Similarly, the specification for \f{plus} in the previous
section cannot distinguish the definition in
\labelcref{eq:plus-def} from:
%
\begin{align}
  \label{eq:plus-def2}
  \f{plus} &= \lambda m.\lambda n.\lambda f.\lambda x.n\,f\,(m\,f\,x)
\end{align}
%
They differ in the order in which the arguments $m$ and $n$
are applied to get the final result.
%
In this case, we can give an alternative specification based
on the evaluation of \f{plus} seen as a recursively-defined
function:
%
\begin{align*}
  \f{?plus}\;0\;y &\equiv y \\
  \f{?plus}\;(\f{succ}\;x)\;y &\equiv \f{succ}\;(\f{?plus}\;x\;y)
\end{align*}
%
where \f{succ} is the successor function defined as:
%
\begin{align*}
  \f{succ} = \lambda n. \lambda f. \lambda x. f\,(n\,f\,x)
\end{align*}

This specification does not just fix the order in which $m$
and $n$ are evaluated but also rule out various spurious
results which satisfy the incomplete specification given in
\cref{sec:io}.
%
However, checking conformance with this sort of
specifications is in general more challenging because it
requires $\eta$-conversion.
%
To see why consider the evaluation of the term in the first
equivalence when we expand \f{plus} with the definition in
\labelcref{eq:plus-def}.
%
Under normal order, $\f{plus}\;0\;y$ evaluates to $\lambda
f.\lambda x. y\,f\,x$ which is only $\eta$-equivalent to
$y$.

% We will be using input-output examples as our specification, so the
%   behavioral constraints of our synthesis problem are simply the
%   evaluation of our examples for a given term or function.
% For instance, the input-output examples to synthesize \textit{and}
%   will look like $?and \, true \, true = true$,
%   $?and \, false \, true = false$,
%   $?and \, true \, false = false$, and
%   $?and \, false \, false = false$, where the $?$ before \texttt{and}
%   signifies the function we are synthesizing.
% Note that because there is no structural difference in untyped lambda
%   calculus between functions and terms, there is no need to differentiate
%   the two in our synthesizer.
% Additionally, we are using a bottom-up search strategy with some
%   special limitations on the form which are described in detail in
%   section \ref{sec:optimizations}.
% We will generate terms from the untyped lambda calculus grammar, with
%   the additional constraint that they must be in normal form.
% Additionally, we will be generating terms using De Bruijn indices rather
%   than with explicit variables names, in order to make evaluation of the
%   terms easier (largely so we don't have to worry about alpha-renaming).
