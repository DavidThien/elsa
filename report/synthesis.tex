\section{Synthesizing Functions for a Given Encoding}
%
As a first step we tackle the problem of synthesizing
functions over a given encoding.
%
We begin with an overview of the standard encoding for the
data types on which we will be focusing.

% The first step we took for synthesizing expressions was to synthesize
%   individual functions given an encoding.

\subsection{Church Encodings}
%
We tested and focused our development on three
different data types: \emph{booleans}, \emph{natural
numbers} and \emph{pairs}.
%
We give their standard encodings below.

\paragraph{Booleans}
%
The standard encoding for the boolean values \emph{true} and
\emph{false} is given by:
%
\begin{align*}
  \text{\true} &= \lambda a . \lambda b . a \\
  \text{\false} &= \lambda a . \lambda b . b
\end{align*}
%
Intuitively, boolean logic is interpreted as a choice and
boolean values are represented as functions of two
parameters making this choice: \emph{true} chooses its first
parameter and \emph{false} chooses its second parameter.

\paragraph{Natural numbers}
%
Church numerals are the standard representation of natural
numbers in the lambda calculus.
%
In this encoding the number $n$ is represented as a
higher-order function that maps a function $f$ to its
$n$-fold composition.
%
The value of the numeral is equivalent to the number of
times the function gets applied: 0 does not apply the
function at all, 1 applies the function once, 2 applies the
function twice, \emph{etc.}
%
\begin{align*}
  0 & = \lambda f . \lambda x . x \\
  1 & = \lambda f . \lambda x . f x \\
  2 & = \lambda f . \lambda x . f (f x) \\
  3 & = \lambda f . \lambda x . f (f (f x)) \\
  \vdots & \\
  n & = \lambda f . \lambda x . f^{\circ n} x
\end{align*}

\paragraph{Pairs}
%
A pair is represented as a function that takes two arguments
and returns a higher-order function which when provided a
function as argument applies it to the two components of the
pair.
%
\begin{align*}
  \pair &= \lambda x. \lambda y. \lambda f.f x y
\end{align*}

% Because there are multiple ways to encode terms in lambda calculus, any
%   function definitions will inherently depend on the encoding.

% We are interested in synthesizing the following functions.

% \begin{align*}
%   & \text{and} = \lambda p . \lambda q . p q p
%   & \text{or} = \lambda p . \lambda q . p p q \\
%   & \text{not} = \lambda p . p (\lambda a . \lambda b . b) (\lambda a . \lambda b . a)
%   & \text{if}  = \lambda p . \lambda a . \lambda b . p a b \\
%   & \text{succ} = \lambda n . \lambda f . \lambda x . f (n f x)
%   & \text{add} = \lambda m . \lambda n . \lambda f . \lambda x . m f ( n f x)
% \end{align*}

\subsection{Synthesis from Examples}
%
We now turn to the question of how we can specify the
behavior of functions for their synthesis.
%
As it have proven useful in many scenarios, we explore the
idea of using input-output examples to specify the behavior
of functions.

Consider, for example, the boolean function \f{and}.
We can specify its behavior extensionally as follows:
%
\begin{align*}
\f{?and}\,\true\,\true    &\equiv \true \\
\f{?and}\,\true\,\false   &\equiv \false \\
\f{?and}\,\false\,\true   &\equiv \false \\
\f{?and}\,\false\, \false &\equiv \false \\
\end{align*}
%
By convention, we prefix the functions we wish to synthesize
with \f{?} and by convenience we refer to the boolean values
\true and \false by name, but they should be understood as
if their definition was expanded.
%
With this specification we can successfully find the
standard definition of \f{and} for church booleans:
\begin{align*}
  \f{and} &= \lambda p. \lambda q. p\,q\,p
\end{align*}

Similarly, we can apply this idea to numerals to specify the
behavior of the \f{plus} function.
%
In this case, however, we cannot specify the behavior on all
possible inputs but we can select a set of representative one.
%
\begin{align*}
  \f{?plus}\; 0\; 1 &\equiv 1\\
  \f{?plus}\; 1\; 0 &\equiv 1\\
  \f{?plus}\; 1\; 2 &\equiv 3\\
  \f{?plus}\; 2\; 2 &\equiv 4\\
\end{align*}
%
The above specification is sufficient for our tool to find
the definition:
%
\begin{align*}
  \f{plus} &= \lambda m. \lambda n. \lambda f. \lambda x. m\,f\,(n\,f\,x)
\end{align*}

% We will be using input-output examples as our specification, so the
%   behavioral constraints of our synthesis problem are simply the
%   evaluation of our examples for a given term or function.
% For instance, the input-output examples to synthesize \textit{and}
%   will look like $?and \, true \, true = true$,
%   $?and \, false \, true = false$,
%   $?and \, true \, false = false$, and
%   $?and \, false \, false = false$, where the $?$ before \texttt{and}
%   signifies the function we are synthesizing.
% Note that because there is no structural difference in untyped lambda
%   calculus between functions and terms, there is no need to differentiate
%   the two in our synthesizer.
% Additionally, we are using a bottom-up search strategy with some
%   special limitations on the form which are described in detail in
%   section \ref{sec:optimizations}.
% We will generate terms from the untyped lambda calculus grammar, with
%   the additional constraint that they must be in normal form.
% Additionally, we will be generating terms using De Bruijn indices rather
%   than with explicit variables names, in order to make evaluation of the
%   terms easier (largely so we don't have to worry about alpha-renaming).
